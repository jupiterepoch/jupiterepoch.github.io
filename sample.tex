\documentclass[a4paper]{article}
\usepackage[colorlinks,linkcolor=black,urlcolor=black]{hyperref}
\usepackage{float}
\usepackage{verbatim}
\usepackage{enumerate}
\usepackage{pdfpages}
\usepackage{listings}
\usepackage{geometry}
\usepackage{multirow}
\usepackage{graphicx}
\usepackage{mathrsfs}
\usepackage{amsmath, amsthm}
\usepackage{amssymb,url}
\usepackage{color}
\usepackage{url}
\usepackage{epsfig}
\usepackage{subfigure}
\usepackage{cite}
% Solutions use a modified proof environment
\newenvironment{solution}
               {\let\oldqedsymbol=\qedsymbol
                \renewcommand{\qedsymbol}{$\blacktriangleleft$}
                \begin{proof}[\bfseries\upshape Solution]}
               {\end{proof}
                \renewcommand{\qedsymbol}{\oldqedsymbol}}
% use \mbox{}\\ to force a change of line
\begin{document}

\section*{Problem 1}
\subsection*{(a)}



\end{document}

%%%%%%%%%%%%%%%%%%%%%%%%%%%%%%%%%%%%%%%%%%%%%%%%%%%%%%%%%%%%%%%%

\begin{equation}
hello
\end{equation}

\begin{figure}[!ht]
    \centering
    \includegraphics[width=100pt]{2.2.a.png}
\end{figure}

\begin{figure}[!ht]
    \centering 
    \subfigure[Edge detection on original image]{ 
    \includegraphics[width=100pt]{gaussian_filter/q3_edge.png}} 
    \hspace{20pt} 
    \subfigure[Edge detection on Gaussian filtered image]{ 
    \includegraphics[width=100pt]{gaussian_filter/q3_edge_gaussian.png}}

    \caption{Comparison} 
\end{figure}

% vertical flip
\scalebox{1}[-1]{\includegraphics[width=0.5\linewidth]{myimg}}
% horizontal flip
\scalebox{-1}[1]{\includegraphics[width=0.5\linewidth]{myimg}}

\definecolor{nblue}{rgb}{0.25, 0.41, 0.88}
\definecolor{gray}{rgb}{0.5,0.5,0.5}
\definecolor{mauve}{rgb}{0.63,0.32,0.18}
\definecolor{dkgreen}{rgb}{0,0.6,0}

% --------------------------------------------------------------
%                         MATLAB code
% --------------------------------------------------------------
% \lstset{language=Matlab,
% keywordstyle=\color{nblue},
% numberstyle=\tiny\color{gray},
% stringstyle=\color{mauve}}

% --------------------------------------------------------------
%                         Python code
% --------------------------------------------------------------
\lstset{frame=tb,
  language=Python,
  aboveskip=3mm,
  belowskip=3mm,
  showstringspaces=false,
  columns=flexible,
  basicstyle={\fontsize{8}{10}},
  numbers=none,
  numberstyle=\tiny\color{gray},
  keywordstyle=\color{nblue},
  commentstyle=\color{dkgreen},
  stringstyle=\color{mauve},
  breaklines=true,
  breakatwhitespace=true,
  tabsize=3
}